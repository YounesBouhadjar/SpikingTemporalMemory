\documentclass[10pt,a4paper,twoside,american]{article}
\usepackage{geometry}
\geometry{verbose,a4paper,tmargin=2cm,bmargin=2cm,lmargin=2cm,rmargin=2cm}
\usepackage{authblk}
\usepackage{calc}
\usepackage{graphicx}
\usepackage[table]{xcolor}
\definecolor{shadecolor}{gray}{0.9}
\usepackage{natbib} 
\usepackage[font=small,labelfont=bf]{caption}
\usepackage{amsmath}
\usepackage{amssymb}
\usepackage{mathtools}
\usepackage{bm}
\usepackage{algorithmic}

\usepackage[%%
breaklinks=true,
colorlinks=true,
linkcolor=gray,citecolor=gray,urlcolor=gray,filecolor=gray,
pdffitwindow,
bookmarks=true,
bookmarksopen=true,
bookmarksnumbered=true,
pdftitle={},
pdfauthor={},
pdfsubject={},
pdfkeywords={}
]
{hyperref}
%%%%%%%%%%%%%%%%%%%%%%%%%%%%%
%%%% change tracking
\usepackage{changes} %% see documentation @ https://ctan.org/pkg/changes
%% define authors and their colors
\definecolor{YB}{RGB}{0,0,200}
\definecolor{TT}{RGB}{0,200,0}
\definecolor{DW}{RGB}{200,0,0}
\definechangesauthor[color=YB]{YB}
\definechangesauthor[color=TT]{TT}
\definechangesauthor[color=DW]{DW}

\newcommand{\note}[2][]{\added[#1]{\footnotesize\it[#2]}} %% add \note command for notes
%%%%%%%%%%%%%%%%%%%%%%%%%%%%%
%%%% crosslinks
\usepackage{cleveref}
\crefformat{equation}{(#1)}
\crefformat{figure}{Fig.\,#1}
\crefformat{section}{Sec.\,#1}
\crefformat{table}{Tab.\,#1}

\usepackage[]{float}
\restylefloat{figure}
%%%%%%%%%%%%%%%%%%%%%%%%%%%%%
%% NEURON specific formating (see author guidelines: https://www.cell.com/neuron/authors)
\renewcommand\familydefault{cmss}      %% NEURON is using sans-serif fonts
\renewcommand{\abstractname}{Summary}  %% abstract is called Summary 

\def\figdir{figures}

%% macros
\newcommand{\panel}[3]{%         
  \parbox[t]{2ex}{\textbf{#1}\,}
  \parbox[t]{#2}{\,\includegraphics[width=#2]{#3}}
}
\newcommand{\pluseq}{\mathrel{+}=}
\newcommand{\subseq}{\mathrel{-}=}
\newcommand*{\ie}{i.e.}
\newcommand*{\eg}{e.g.}
\newcommand*{\etc}{etc.}
\newcommand*{\etal}{{et al.}}
\newcommand*{\cf}{{cf.}}
\newcommand*{\ca}{c.}
\newcommand*{\vs}{vs}
\newcommand*{\amp}{\&}
\newcommand*{\bd}{\hspace{0pt}}%
\providecommand*{\urlalt}{\href}
\newcommand{\wmin}{w_\text{min}}
\newcommand*{\iselement}{\in}
\newcommand*{\whij}{w_{ij}^{\text{ho}}}
\newcommand*{\synweight}{J}
\newcommand*{\ADhomeo}{\text{ADH}}
\newcommand*{\weightextstim}{J_{\text{ext}}}
\newcommand*{\jitterstar}{\alpha^{*}}
\newcommand*{\rankstar}{R^{*}}
\newcommand*{\jitter}{\alpha}
\newcommand*{\rank}{r}
\newcommand*{\tobs}{t_{\text{obs}}}
\newcommand*{\tp}{t_{\text{p}}}
\newcommand*{\lmax}{\lambda_{\text{max}}}
\newcommand*{\lmaxADH}{\lambda_{\text{max}_\text{ADH}}}
\newcommand*{\spikeresponse}{s}
\newcommand*{\spike}[2][]{{s}^{#1}_{#2}(t)}
\newcommand*{\lpass}[2][]{{x}^{#1}_{#2}(t)}
\newcommand*{\pearsoncorrelationcoeff}{C}
\newcommand*{\spiketrains}{x}
\newcommand*{\simtimePC}{t_{cc}}
\newcommand*{\readouttime}{t_\text{obs}}
\newcommand*{\numstims}{p}
\newcommand*{\numtemplates}{k}
\newcommand*{\numtrials}{m}
\newcommand*{\numneurons}{N}
\newcommand*{\firingrate}{\nu}
\newcommand*{\firingrateins}{F}
\newcommand*{\effweights}{w}
\newcommand*{\effweightsMat}{W}
\newcommand*{\flucsgroundstate}{z(t)}
\newcommand*{\tauht}{\tau}
\newcommand*{\y}{y}
\newcommand*{\resetpot}{r}
\newcommand*{\spikingthr}{\theta}
\newcommand*{\inputflucs}{\sigma}
\newcommand*{\inputmean}{\mu}
\newcommand*{\workingpoint}{*}

%%%%%%%%%%%%%%%%%%%%%%%%%%%%%%%%%%%
%% sorted alphabetically
\newcommand{\CM}{C_\textnormal{m}}    %% membrane time constant
\newcommand{\CV}{\textnormal{CV}}     %% coefficient of variation
\newcommand{\dtfilter}{\Delta t_\textnormal{f}}
\newcommand{\dtpert}{\delta t^*}
\newcommand{\dtsim}{\Delta t}
\newcommand{\DS}{\textnormal{DS}}
\newcommand{\EE}{{\exc\exc}}
\newcommand{\EI}{{\exc\inh}}
\newcommand{\EF}{{\exc\textnormal{F}}}
\newcommand{\exc}{\textnormal{E}}     %% label for ``excitatory''
\newcommand{\ext}{\textnormal{X}}   %% label for ``external''
\renewcommand{\exp}{\textnormal{exp}} %% exponential function
\newcommand{\erf}{\textnormal{erf}} %% error function
\newcommand{\EW}[2][]{\left\langle{#2}\right\rangle_{#1}}
\newcommand{\FF}{\textnormal{FF}}     %% Fano factor
\newcommand{\IE}{{\inh\exc}}
\newcommand{\II}{{\inh\inh}}
\newcommand{\Epop}{\mathcal{E}} %% set (hence caligraphic) of excitatory neurons
\newcommand{\inh}{\textnormal{I}}     %% label for ``inhibitory''
\newcommand{\Ipop}{\mathcal{I}} %% set (hence caligraphic) of inhibitory neurons
\newcommand{\Imag}{\textnormal{Im}}
\newcommand{\PSCamp}{\hat{I}}               %% PSC amplitude (pA)
\newcommand{\J}{J}                          %% PSP amplitude (mV)
\newcommand{\JDS}{J_\textnormal{DS}}
\newcommand{\JSD}{J_\textnormal{SD}}
\newcommand{\JEF}{J_\textnormal{EF}}
\newcommand{\JE}{{\J_{\exc}}}
\newcommand{\JEE}{\J_{\exc\exc}}
\newcommand{\JEI}{\J_{\exc\inh}}
\newcommand{\JI}{\J_{\inh}}
\newcommand{\JIE}{\J_{\inh\exc}}
\newcommand{\JII}{\J_{\inh\inh}}
\newcommand{\JX}{\J_\ext}
\newcommand{\Jfb}{\hat{J}}                   %% synapic weights (in mV) as defined in Fourcaud & Brunel (2002)
\newcommand{\JfbE}{\Jfb_{\exc}}
\newcommand{\JfbEE}{\Jfb_{\exc\exc}}
\newcommand{\JfbEI}{\Jfb_{\exc\inh}}
\newcommand{\JfbI}{\Jfb_{\inh}}
\newcommand{\JfbIE}{\Jfb_{\inh\exc}}
\newcommand{\JfbII}{\Jfb_{\inh\inh}}
\newcommand{\JfbX}{{\Jfb_\ext}}
\newcommand{\K}{K}
\newcommand{\KD}{K_{\textnormal{D}}}
\newcommand{\KE}{K_{\exc}}
\newcommand{\KEE}{{K_{\exc\exc}}}
\newcommand{\KEI}{{K_{\exc\inh}}}
\newcommand{\KI}{{K_{\inh}}}
\newcommand{\KIE}{{K_{\inh\exc}}}
\newcommand{\KII}{{K_{\inh\inh}}}
\newcommand{\Kinput}{{K^\textnormal{out}_\ext}} %% outdegree of each inpu sources
\newcommand{\KX}{{K_\ext}} %% number of inputs from external sources (external in-degree)
\newcommand{\mat}[1]{\bm{#1}}
\renewcommand{\max}{\textnormal{max}} %% exponential function
\newcommand{\ms}{\,\textnormal{ms}}
\newcommand{\muE}{{\mu_\exc}}
\newcommand{\muI}{{\mu_\inh}}
\newcommand{\mV}{\,\textnormal{mV}}
\newcommand{\ND}{{N_{\textnormal{D}}}}
\newcommand{\NE}{{N_{\exc}}}
\newcommand{\NI}{{N_{\inh}}}
\newcommand{\nE}{{n_\exc}}
\newcommand{\nI}{{n_\inh}}
\newcommand{\nuE}{\nu_\exc}
\newcommand{\nuI}{\nu_\inh}
\newcommand{\nuX}{\nu_\ext}
\newcommand{\pA}{\,\textnormal{pA}}
\newcommand{\pert}{\textnormal{pert}}
\newcommand{\pF}{\,\textnormal{pF}}
\newcommand{\Real}{\textnormal{Re}}
\newcommand{\RM}{R_\textnormal{m}}
\newcommand{\SD}{\textnormal{SD}}
\newcommand{\seconds}{\,\textnormal{s}}
\newcommand{\sigmaE}{\sigma_{\exc}}
\newcommand{\sigmaI}{\sigma_{\inh}}
\newcommand{\sps}{\,\textnormal{spikes/s}}
\newcommand{\syn}{\textnormal{s}}
\newcommand{\taufilter}{\tau_\textnormal{f}}
\newcommand{\tauM}{\tau_\textnormal{m}}
\newcommand{\tauR}{\tau_\textnormal{ref}}
\newcommand{\tauS}{\tau_\syn}
%\renewcommand{\transp}{^\textnormal{T}}
\newcommand*{\Var}{\textnormal{Var}}
\renewcommand{\vec}[1]{\bm{#1}}
\newcommand{\Vreset}{V_\textnormal{r}}
\newcommand{\VD}{V_\textnormal{D}}
\newcommand{\VS}{V_\textnormal{S}}
\newcommand{\VI}{V_\textnormal{I}}
\newcommand{\effweight}{w}                   %% synapic weights (in mV) as defined in Fourcaud & Brunel (2002)
\newcommand{\wE}{\effweight_{\exc}}
\newcommand{\wEE}{\effweight_{\exc\exc}}
\newcommand{\wEI}{\effweight_{\exc\inh}}
\newcommand{\wI}{\effweight_{\inh}}
\newcommand{\wIE}{\effweight_{\inh\exc}}
\newcommand{\wII}{\effweight_{\inh\inh}}
\newcommand{\yreset}{y_\textnormal{r}}
\newcommand{\ytheta}{y_\theta}

% defining the \BibTeX command - from Oren Patashnik's original BibTeX documentation.
\def\BibTeX{{\rm B\kern-.05em{\sc i\kern-.025em b}\kern-.08emT\kern-.1667em\lower.7ex\hbox{E}\kern-.125emX}}

%%%%%%%%%%%%%%%%%%%%%%%%%%%%%%%%%%%%%%%%%%%%%%%%%%%%%%%%%%%%%%%%%%%%%%%%%%%%%%%%%%%%%%%%%%%%%%%%%%%%%%%%%%%%%%%%%
    
\begin{document}

\title{Model description:\\{\bf ExtSpikingTM}}
\author{}
\date{}
\maketitle
\thispagestyle{empty}

%%%%%%%%%%%%%%%%%%%%%%%%%%%%%%%%%%%%%%%%%%%%%%%%%%%%%%%%%%%%%%%%%%%%
\section{Network model}
\label{sec:suppl_network_model}

%%%%%%%%%%%%%%%%%%%%%%%%%%%%%%%%%%%%%%%%%%%%%%%%%%%%%%%%%%%%%%%%%%%%%%%%%%%%%%%%%%%%
\begin{table}[H]
\renewcommand{\arraystretch}{1.1}
%%%%%%%%%%%%%%%%%%%
\begin{tabular}{|@{\hspace*{1mm}}p{3cm}@{}|@{\hspace*{1mm}}p{12cm}|}
\hline 
\multicolumn{2}{|>{\color{white}\columncolor{black}}c|}{\textbf{Summary}}\\
\hline
\textbf{Populations} &  excitatory ($\Epop$), inhibitory ($\Ipop$) and external (X) \\
\hline 
\textbf{Connectivity} &
\begin{itemize}
    \item sparse random connectivity between excitatory neurons (plastic)
    \item local recurrent connectivity between excitatory and inhibitory neurons (static)
    % \item excitatory to excitatory ($\EE$): random connection (from soma to the dendritic compartment of other neurons)
    %\item inhibitory to excitatory ($\EI$): all-to-all (from interneuron to soma).
    %\item excitatory to inhibitory ($\IE$): all-to-all (from soma to interneuron). 
    % \item external stimulus to excitatory: connects directly to the soma of excitatory neurons.  
\end{itemize}
\\
\hline
\textbf{Neuron model} & 
\begin{itemize}
\item excitatory neurons: point neuron (LIF) with nonlinear summation of inputs      
\item inhibitory neurons: point neuron (LIF)
\end{itemize}
\\
\hline 
\textbf{Synapse model } & additive (linear) exponential postsynaptic currents (PSCs)  \\
\hline 
\textbf{Plasticity } &  spike-timing dependent structural plasticity and firing-rate homeostasis in excitatory to excitatory connections
\\
\hline
\end{tabular}
%%%%%%%%%%%%%%%%%%%
\caption{Description of the network model (continued on next page).}
\label{tab:model_description}
\end{table}
%%%%%%%%%%%%%%%%%%%%%%%%%%%%%%%%%%%%%%%%%%%%%%%%%%%%%%%%%%%%%%%%%%%%%%%%%%%%%%%%%%%%
%% model description table (continued)
\addtocounter{table}{-1}
\begin{table}[H]
%%%%%%%%%%%%%%%%%%%
%%%%%%%%%%%%%%%%%%% 
\begin{tabular}{|@{\hspace*{1mm}}p{3cm}@{}|@{\hspace*{1mm}}p{5.95cm}@{}|@{\hspace*{1mm}}p{5.95cm}|}
\hline 
\multicolumn{3}{|>{\color{white}\columncolor{black}}c|}{\textbf{Populations}}\\
\hline
\multicolumn{3}{|l|}{Ensemble of subpopulations $\{i|i=1, \ldots, M  \}$}\\
\hline 
\textbf{Name} & \textbf{Elements} & \textbf{Size}\\
\hline 
$\Epop_{i}$ & excitatory neurons in subpopulation $i$ & $n_\exc$ \\
\hline 
$\Ipop_{i}$ & inhibitory neurons in subpopulation $i$ & $n_\inh$ \\
\hline 
\end{tabular}
%%%%%%%%%%%%%%%%%%%
\begin{tabular}{|@{\hspace*{1mm}}p{1.45cm}@{}|@{\hspace*{1mm}}p{1.45cm}@{}|@{\hspace*{1mm}}p{12cm}|}
\hline 
\multicolumn{3}{|>{\color{white}\columncolor{black}}c|}{\textbf{Connectivity}}\\
\hline 
\textbf{Source} & \textbf{Target} & \textbf{Pattern, same in the parameter table }\\
\hline 
$\Epop_{i}$ ($\forall{i}$) & $\Epop_{j}$ ($\forall{j}$) & random, fixed in-degrees $K_\EE$ and delays $d_{\EE}$, plastic weights $J_{\EE, ji}$  ($\EE$ connections) \\
\hline 
$\Epop_{i}$ ($\forall{i}$)  & $\Ipop_{i}$ ($\forall{i}$) & all-to-all, fixed delays $d_{\IE}$, fixed weights $J_\IE$ (suprathreshold) ($\IE$ connections) \\
\hline 
$\Ipop_{i}$ ($\forall{i}$)& $\Epop_{i}$ ($\forall{i}$)& all-to-all, fixed delays $d_{\EI}$, fixed weights $J_\EI$ ($\EI$ connections) \\
\hline 
$\Ipop_{i}$ ($\forall{i}$)& $\Ipop_{i}$ ($\forall{i}$)& none ($\II$ connections) \\
\hline
\multicolumn{3}{|l|}{no self-connections (``autapses''), no multiple connections (``multapses'') }\\
\hline 

\end{tabular}
%%%%%%%%%%%%%%%%%%%
\caption{Description of the network model (continued on next page).}
\label{tab:Model-description} 
\end{table}
%%%%%%%%%%%%%%%%%%%%%%%%%%%%%%%%%%%%%%%%%%%%%%%%%%%%%%%%%%%%%%%%%%%%%%%%%%%%%%%%%%%%
%%%%%%%%%%%%%%%%%%%%%%%%%%%%%%%%%%%%%%%%%%%%%%%%%%%%%%%%%%%%%%%%%%%%%%%%%%%%%%%%%%%% 
\setcounter{table}{\thetable-1}
\begin{table}[ht!]
%%%%%%%%%%%%%%%%%%%
\begin{tabular}{|@{\hspace*{1mm}}p{3cm}@{}|@{\hspace*{1mm}}p{12cm}|}
  \hline 
  \multicolumn{2}{|>{\color{white}\columncolor{black}}c|}{\textbf{Neuron and synapse}}\\
  %%%%%%%%%%%%%
  \hline
  \multicolumn{2}{|>{\columncolor{lightgray}}c|}{
  \textbf{Excitatory neurons}
  }\\
  \hline
  \textbf{Type} & point neuron (LIF) with nonlinear dendritic input integration \\
  \hline
    \textbf{Description} & dynamics of membrane potential $V_{i}(t)$ of neuron $i\in[1,N_\exc]$:                
               \begin{itemize}
               \item emission of the $k$th spike of neuron $i$ at time $t_{k}^{i}$ if $V_{i}(t_{k}^{i})\geq\theta$ %and $V_{i}\left(t_{k-1}^{i}\right)\leq\theta$
               \item subthreshold dynamics:
                 \begin{equation*}
                     \tauM\dot{V}_{i} =-V_{i}+\RM\left(I_{\text{X},i}(t) + I_{\text{I},i}(t)  + \sum_{q=1}^{\ND} I_{\text{D},iq}(t) \right)
                     \quad \forall t \notin \left[t_{k}^{i},\,t_{k}^{i}+\tauR\right)
                 \end{equation*}
                 with $I_{\text{D},iq}(t)$ ($q\in[1,\ND]$) the synaptic currents from $\ND$ dendritic branches, $I_{\text{X},i}(t)$ the external input, and $I_{\text{I},i}(t)$ the inhibitory input.
                  \item resets and refractoriness:
                  \begin{equation*}
                     V_{i}(t)=\Vreset
			  \quad \forall{}k,\ \forall t \in \left[t_{k}^{i},\,t_{k}^{i}+\tauR\right]
                 \end{equation*}
               \item  synaptic currents dynamics of $I_{\text{D},iq}$, $I_{\text{X},i}$ and $I_{\text{I},i}$ :
               \begin{equation*}
                 \tauS\dot{I}_{i}(t) = -I_{i}(t) + \sum_{j=1}^{N} \J_{ij} \cdot s_j(t)
               \end{equation*}
               \item presynaptic spike trains $s_j(t)=\sum_k \delta(t-t_k^j)$
               \item emission of dAP in dendritic branch $q$ at times $t_{l}^{iq}$:    
               \begin{equation*}
                  I_{\text{D},iq}(t) = I_\text{P}\quad  
                  if \     I_{\text{D},iq}(t_{l}^{iq})\geq\theta_{\text{dend}}\quad 
                  \forall{}l,\ \forall t \in \left(t_{l}^{iq},\,t_{l}^{iq}+\tau_\text{dend}\right]    
               \end{equation*}
               Where $I_\text{p}$ is the current plateau of the dAP with its duration $\tau_\text{dend}$ and activation threshold $\theta_{\text{dend}}$
                
               \item dAP rate $\nu_{\text{d}}$:   
               \begin{equation*}
                 \tau_\text{d}\dot{\nu}_{\text{d}, i} = -\nu_{\text{d}, i}(t) + d_i(t)
               \end{equation*}
               Where $\tau_\text{d}$ is the time constant and $d_i(t)=\sum_i \delta(t-t_l^i)$, which comprises the dAP generation times.        
   
              \end{itemize}\\
  %%%%%%%%%%%%%
  \hline
      \multicolumn{2}{|>{\columncolor{lightgray}}c|}{
    \textbf{Inhibitory neurons}
  }\\
  \hline
  \textbf{Type} & point neuron (LIF) \\
  % \hline
  % \textbf{Model } &
  %                   \begin{itemize}
  %                   \item excitatory neurons: multi-compartment model with one soma compartment (LIF) and $\ND$ dendritic branches, each described by one compartment (LIF)
  %                   \item inhibitory neurons: point neuron (LIF)

  %                   \end{itemize}                     
  %                   \\
                    \hline
  \textbf{Description} & dynamics of membrane potential $V_{c}(t)$ of compartment $c$:                 
                       \begin{itemize}
                       \item spike emission at time $t_{k}^{c}$ if $V_{c}(t_{k}^{c})\geq\theta$ %and $V_{i}\left(t_{k-1}^{i}\right)\leq\theta$ 
                       \item subthreshold dynamics: \begin{equation*}\label{eq:1}
                        \tauM\dot{V}_{c}=-V_{c}+\RM{}I_{c}(t) \quad \forall{}k,\ \forall t \notin \left[t_{k}^{c},\,t_{k}^{c}+\tauR\right) \end{equation*}
                       \item reset and refractoriness: 
                         $V_{c}(t)=\Vreset$ \quad $\forall{}k,\ \forall t \in \left(t_{k}^{c},\,t_{k}^{c}+\tauR\right]$  
                        \item synaptic currents dynamics: 
                           \begin{equation*}
                             \tauS\dot{I}_c = -I_c(t) + \sum_{j=1}^{N} \J_{ij}  \cdot s_j(t)
                           \end{equation*}
                         
                         % if for all $k:$ $t\in$$\left(t_{k}^{i},\,t_{k}^{i}+\tau_{ref}\right]$
                       \end{itemize}
                     \\
                  \hline 
\end{tabular}
%%%%%%%%%%%%%%%%%%%
\caption{Description of the network model (continued on next page).}
\end{table}
%%%%%%%%%%%%%%%%%%%%%%%%%%%%%%%%%%%%%%%%%%%%%%%%%%%%%%%%%%%%%%%%%%%%%%%%%%%%%%%%%%%%
%%%%%%%%%%%%%%%%%%%%%%%%%%%%%%%%%%%%%%%%%%%%%%%%%%%%%%%%%%%%%%%%%%%%%%%%%%%%%%%%%%%% 
\setcounter{table}{\thetable-1}
\begin{table}[ht!]
  %%%%%%%%%%%%%%%%%%% 
  \begin{tabular}{|@{\hspace*{1mm}}p{3cm}@{}|@{\hspace*{1mm}}p{12.cm}|}
  \hline 
  \multicolumn{2}{|>{\color{white}\columncolor{black}}c|}{\textbf{Plasticity}}\\                      \hline
  \textbf{EE synapses} &
                         \begin{itemize}                                                                                        \item spike-timing dependent structural plasticity and firing-rate homeostasis
                         \item each synapse described by permanence $P_{ij}$ (maturity) and synaptic weight $\J_{\EE, ij}$                                                                                                 
                         \item update $P_{ij}$ and $J_{\EE, ij}$ at time $t^k$ of the $k$th spike of presynaptic neuron $j$:
      \begin{algorithmic}

        \STATE  $L_\text{post} \gets$ list of postsynaptic spike times $t^i$ in interval $(t^j_{k-1},t^j_{k}]$      
        
        \FOR{$t^i$ in $L_\text{post}$}
        \STATE $P_{ij} \pluseq \lambda \cdot \exp(- |t^i - t^j_{k-1}|/\tau_{+})$
        (potentiation)
        \STATE $P_{ij} \pluseq \kappa \cdot (\nu_\text{d}^* - \nu_\text{d})$ (homeostasis)
    \ENDFOR           
    
    \STATE $P_{ij} \subseq l$  (depression [leak])
  
    \IF{$ P_{ij} > P_\text{max} $}
      \STATE $P_{ij} = P_\text{max}$ (clipping at maximum permanence)
    \ENDIF
    
    \IF {$ P_{ij} > P_{\theta} $} 
       \STATE $\J_{\EE, ij} \gets W $ (mature synapse)
    \ELSE
       \STATE $\J_{\EE, ij} \gets 0 $ (immature synapse)
    \ENDIF
  \end{algorithmic}
  %here, $T_\text{post}$ denotes the spike timings of the postsynaptic neuron. 

\end{itemize}

  \\
  \hline 
    \textbf{all other synapses} & 
                   non-plastic
  \\
  \hline
\end{tabular}
%%%%%%%%%%%%%%%%%%%
\begin{tabular}{|@{\hspace*{1mm}}p{15.15cm}|}
%\hline 
  \multicolumn{1}{|>{\color{white}\columncolor{black}}c|}{\textbf{Input}}\\
  \begin{itemize}
  \item $N_\text{stim}$ spike-generating sources, each representing a character in the sequences
  \item each spike source connected to a random subset of $L$ subpopulations via a synaptic weight $J_\EF$ and a delay of $d_\EF$
  \end{itemize}
  \\
\hline 
\end{tabular}
%%%%%%%%%%%%%%%%%%%
\begin{tabular}{|@{\hspace*{1mm}}p{15.15cm}|}
  \multicolumn{1}{|>{\color{white}\columncolor{black}}c|}{\textbf{Training}}\\
  \begin{itemize}
  \item repetitive stimulation of network ($k$ times) using the same batch of data.
  \item a batch of data contains S sequences, each sequence is composed of C characters
  \end{itemize}
  \\
  \hline
\end{tabular}
%%%%%%%%%%%%%%%%%%% 
\begin{tabular}{|@{\hspace*{1mm}}p{15.15cm}|}
  \multicolumn{1}{|>{\color{white}\columncolor{black}}c|}{\textbf{Initial conditions}} \\
  \begin{itemize}
    \item for every network realization: initial permanences (at time $0$) randomly drawn from uniform distributions in the interval $[P_\text{min},P_0)$. Membrane potentials are initialized to $V_\text{r}$.
  \end{itemize}\\
  \hline
\end{tabular}
%%%%%%%%%%%%%%%%%%%
\begin{tabular}{|@{\hspace*{1mm}}p{15.15cm}|}
\multicolumn{1}{|>{\color{white}\columncolor{black}}c|}{\textbf{Simulation details}}\\
\hline
\begin{itemize}
    \item network simulations performed in NEST version \note[id=YB]{2.20.0} \citep{nest2200}
    \item synchronous update using exact integration of the system dynamics on a discrete time grid with step size $\dtsim$ \citep{Rotter99a}
\end{itemize}
\\
\hline
\end{tabular}
%%%%%%%%%%%%%%%%%%%
\caption{Description of the network model (continued).}
\end{table}

\clearpage

\subsection{Model and simulation parameters}
\label{app:suppl_parameters}

%%%%%%%%%%%%%%%%%%%%%%%%%%%%%%%%%%%%%%%%%%%%%%%%%%%%%%%%%%%%%%%%%%%%%%%%%%%%%%%%%%%%
\begin{table}[ht!]
\renewcommand{\arraystretch}{1.2}
\begin{tabular}{|@{\hspace*{1mm}}p{3cm}@{}|@{\hspace*{1mm}}p{4cm}@{}|@{\hspace*{1mm}}p{8.1cm}|}
\hline
\textbf{Name} & \textbf{Value} & \textbf{Description}\\
\hline                               
\multicolumn{3}{|>{\columncolor{lightgray}}c|}{\textbf{Network}}\\
%\hline 
%\textbf{Name} & \textbf{Value} & \textbf{Description}\\
\hline 
$N_\exc$ & $3900$ & number of excitatory neurons \\
\hline
$N_\inh$ & $26$ & number of inhibitory neurons \\
\hline
$M$ & $26$ & number of subpopulations \\
\hline
$\nE$,$\nI$ & $150$,$1$ & number of excitatory and inhibitory neurons per subpopulation \\
\hline
\multicolumn{3}{|>{\columncolor{lightgray}}c|}{\textbf{(Potential) Connectivity}}\\
%\hline 
%\textbf{Name} & \textbf{Value } & \textbf{Description}\\
\hline 
$\KEE$ & $270$ & number of excitatory inputs per excitatory neuron ($\EE$ in-degree) \\
\hline 
$\KEI$ & $1$ & number of inhibitory inputs per excitatory neuron ($\EI$ in-degree) \\
\hline 
$\KIE$ & $\nE$ & number of excitatory inputs per inhibitory neuron ($\IE$ in-degree) \\
\hline 
$\KII$ & $0$ & number of inhibitory inputs per inhibitory neuron ($\II$ in-degree) \\
%\hline 
%\end{tabular}
%%%%%%%%%%%%%%%%%%%%%%%%%%%%%%%%%%%%%%%%%%%%%
%\begin{tabular}{|@{\hspace*{1mm}}p{3cm}@{}|@{\hspace*{1mm}}p{4cm}@{}|@{\hspace*{1mm}}p{8.1cm}|}
\hline 
\multicolumn{3}{|>{\columncolor{lightgray}}c|}{\textbf{Excitatory neurons}}\\
\hline 
$\tauM$ & $10\ms$ & membrane time constant \\
\hline 
$\tauR$ & $20\ms$ & absolute refractory period \\
\hline 
$\CM$ & $250\pF$ & membrane capacity \\
\hline 
$\Vreset$ & $0.0\mV$ & reset potential \\
\hline 
$\theta$ & $20\mV$ & somatic spike threshold \\
\hline 
$ I_\text{p}$ & $200\pA$ &  dAP current plateau amplitude\\
\hline 
$\tau_\text{dend}$ & $60\ms$ & dAP duration\\
\hline 
$\theta_{\text{dend}}$ & $60\pA$ & dAP threshold \\
%\hline
%\end{tabular}
%\begin{tabular}{|@{\hspace*{1mm}}p{3cm}@{}|@{\hspace*{1mm}}p{4cm}@{}|@{\hspace*{1mm}}p{8.1cm}|}
%%%%%%%%%%%%%%%%%%%%%%%%%%%%%%%%%%%%%%%%%%%%%
\hline 
\multicolumn{3}{|>{\columncolor{lightgray}}c|}{\textbf{Inhibitory neurons}}\\
\hline
$\tauM$ & $5\ms$ & membrane time constant\\
\hline 
$\tauR$ & $2\ms$ & absolute refractory period\\
\hline 
$\CM$ & $250\pF$ & membrane capacity\\
\hline 
$\Vreset$ & $0.0\mV$ & reset potential\\
\hline 
$\theta$ & $15\mV$ & spike threshold\\
\hline
\end{tabular}
%%%%%%%%%%%%%%%%%%%
\caption{Model and simulation parameters (continued on next page).}
\label{tab:Model-parameters} 
\end{table}
%%%%%%%%%%%%%%%%%%%%%%%%%%%%%%%%%%%%%%%%%%%%%%%%%%%%%%%%%%%%%%%%%%%%%%%%%%%%%%%%%%%%
%%%%%%%%%%%%%%%%%%%%%%%%%%%%%%%%%%%%%%%%%%%%%%%%%%%%%%%%%%%%%%%%%%%%%%%%%%%%%%%%%%%% 
\setcounter{table}{\thetable-1}
\begin{table}[ht!]
%%%%%%%%%%%%%%%%%%%
\begin{tabular}{|@{\hspace*{1mm}}p{3cm}@{}|@{\hspace*{1mm}}p{4cm}@{}|@{\hspace*{1mm}}p{8.1cm}|}
\hline
\textbf{Name} & \textbf{Value } & \textbf{Description}\\
\hline
\multicolumn{3}{|>{\columncolor{lightgray}}c|}{\textbf{Synapse}}\\
%\hline 
%\textbf{Name} & \textbf{Value } & \textbf{Description}\\
\hline 
$\J_{\EE, ij}$ & $\{0,W\}$ & E-neuron $\to$ E-neuron connections strength (PSP amplitude of E-neuron $i$ evoked by a spike from other E-neuron $j$)\\
\hline
$W$ & $12\mV$ & PSP amplitude of the E-neuron evoked by a spike from other E-neuron if the synapse is mature \\
\hline
$\JIE$ & $0.75\mV$ & E-neuron $\to$ I-neuron connections strength (PSP amplitude of the I-neuron evoked by a spike from E-neuron), this determines the pattern size: $s=\frac{\theta}{\JIE}$\\
\hline
$\JEI$ & $-22.5 \mV$ & I-neuron $\to$ E-neuron connections strength (PSP amplitude of the E-neuron evoked by a spike from I-neuron) \\
\hline 
${\tauS}_{,\EE}$ & $5 \ms$ & time constant of E-neuron $\to$ E-neuron connection (post-synaptic current)\\
\hline 
${\tauS}_{,\IE}$ & $0.5 \ms$ & time constant of E-neuron $\to$ I-neuron connection (post-synaptic current) \\
\hline 
${\tauS}_\EI$ & $1 \ms$ & time constant of I-neuron $\to$ E-neuron connection (post-synaptic current) \\
\hline
$d_{\EE}$ & $2\ms$ & delay of E-neuron $\to$ E-neuron connection \\
\hline
$d_\IE$ & $0.1\ms$ & delay of E-neuron $\to$ I-neuron connection \\
\hline
$d_\EI$ & $0.1\ms$ & delay of I-neuron $\to$ E-neuron connection \\
\hline 
\multicolumn{3}{|>{\columncolor{lightgray}}c|}{\textbf{Plasticity}}\\
\hline
$P_{ij}$ & $[P_\text{min},P_0]$  & synaptic permanence \\
\hline
$\lambda$ & 0.01 & learning rate \\
\hline
$l$ & 0.01 & leak term (permanence decrement) \\
\hline
$P_{\theta}$ & 20 & synapse maturity threshold
% when the permanence reaches this value the plastic weight ($\JEE$) becomes W
                            \\
\hline
$P_\text{max}$ & 50 & permanence upper bound\\
\hline
$P_\text{min}$ & 0 & permanence lower bound  \\
\hline
$P_0$ & 8 & permanence initialization upper bound \\
\hline                        
$ \tau_{+} $ & 20 \ms & plasticity time constant (potentiation) \\
\hline                        
$ \nu_\text{d}^* $ & 1 & target dAP rate \\
\hline
$ \kappa $ & 0.2 & homeostasis factor \\
\hline                                              
\multicolumn{3}{|>{\columncolor{lightgray}}c|}{\textbf{Input}}\\
%\hline 
%\textbf{Name} & \textbf{Value } & \textbf{Description}\\
\hline 
$N_\text{stim}$ & $26$ & number of external spike sources \\
\hline 
$L$ & $1$ & number of subpopulations connected to the same spike source \\
\hline
$S$ & for task I: 2, for task II: 5 & number of sequences per batch \\
\hline 
$C$ & for task I: 4, for task II: 6 & number of characters per sequence \\
\hline
$A$ & 26 & length of the total alphabet \\
%\hline 
%\end{tabular}
%%%%%%%%%%%%%%%%%%%%%%%%%%%%%%%%%%%%%%%%%%%%%
%\begin{tabular}{|@{\hspace*{1mm}}p{3cm}@{}|@{\hspace*{1mm}}p{4cm}@{}|@{\hspace*{1mm}}p{8.1cm}|}
\hline 
$\JEF$ & $22 \mV$ & Feedforward input $\to$ E-neuron connections strength (PSP amplitude of E-neuron $i$ evoked by external spike) \\
\hline 
${\tauS}_{,\EF}$ & $2 \ms$ & time constant of external input $\to$ E-neuron connection (post-synaptic current) \\
\hline 
$d_\text{EF}$ & $0.1\ms$ & delay of external input $\to$ E-neuron connection \\
\hline
\multicolumn{3}{|>{\columncolor{lightgray}}c|}{\textbf{Simulation}}\\
%\hline 
%\textbf{Name} & \textbf{Value } & \textbf{Description}\\
\hline
$\Delta T$ & [2, 90] \ms & interstimulus timing \\
\hline
$\dtsim$ & $0.1\ms$ & time resolution \\
\hline
$k$ & [$100$, $400$] & number of training episodes  \\
\hline 
\end{tabular}
%%%%%%%%%%%%%%%%%%%%%%%%%%%%%%%%%%%%%%%%%%%%%
\caption{Model and simulation parameters (continued).}
\end{table}


%%%%%%%%%%%%%%%%%%%%%%%%%%%%%%%%%%%%%%%%%%%%%%%%%%%%%%%%%%%%%%%%%%%%
% add bibliography 
\begin{thebibliography}{}

\bibitem[\protect\citeauthoryear{Linssen et~al.}{Linssen
  et~al.}{2018}]{Nest2160}
Linssen, C., Lepper{\o}d, M.~E., Mitchell, J., Pronold, J., Eppler, J.~M.,
  Keup, C., Peyser, A., Kunkel, S., Weidel, P., Nodem, Y., Terhorst, D., Deepu,
  R., Deger, M., Hahne, J., Sinha, A., Antonietti, A., Schmidt, M., Paz, L.,
  Garrido, J., Ippen, T., Riquelme, L., Serenko, A., K\"{u}hn, T., Kitayama,
  I., M{\o}rk, H., Spreizer, S., Jordan, J., Krishnan, J., Senden, M., Hagen,
  E., Shusharin, A., Vennemo, S.~B., Rodarie, D., Morrison, A., Graber, S.,
  Schuecker, J., Diaz, S., Zajzon, B., \& Plesser, H.~E. (2018).
\newblock Nest 2.16.0.

\bibitem[\protect\citeauthoryear{Rotter \& Diesmann}{Rotter \&
  Diesmann}{1999}]{Rotter99a}
Rotter, S., \& Diesmann, M. (1999).
\newblock Exact digital simulation of time-invariant linear systems with
  applications to neuronal modeling.
\newblock {\em Biol. Cybern.\/}~{\em 81\/}(5-6), 381--402.

\end{thebibliography}

\end{document}

%%% Local Variables:
%%% mode: latex
%%% TeX-master: t
%%% End:


\end{document}
